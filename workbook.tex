% !TeX spellcheck = en_US
\documentclass[12pt,a4paper,oneside]{book}  
% oneside or openright

%% ***************************************************************
%%    PACKAGES
\usepackage{lmodern}         % font package.
\usepackage[T1]{fontenc}     % define T1 charset for out files.
\usepackage[english]{babel}  % italian latex typo conventions.
\usepackage[utf8]{inputenc}  % italian symbols.
\usepackage{csquotes}        % needed by babel.
\usepackage{amsmath}         % math features.
\usepackage{amsthm}          % math theorems.
\usepackage{amssymb}         % math symbols.
\usepackage{amsbsy}			 % math bold.
\usepackage{listings}        % embed programming language in latex.
\usepackage{stmaryrd}        % symbols for theoretical computer science.
\usepackage{hhline}          % better horizontal lines in tabulars and arrays.
%%\usepackage{vmargin}         % various page dimensions.
\usepackage{hyperref}        % hypertext support.
\usepackage{makeidx}         % for creating indexes.
\usepackage{nicefrac}        % inline fractions.
\usepackage{marginnote}      % notes in the margin, even where \marginpar fails.
\usepackage{xr}              % references to other latex documents.
\usepackage{subfiles}        % multifile support.
\usepackage{geometry}        % interface for document dimension.
\usepackage{graphicx}        % enhanced support for graphics.
\usepackage{fancyhdr}        % extensive control of page headers and footers.
\usepackage{lipsum}          % generate dummy text.
\usepackage[
backend=biber,
style=numeric,
citestyle=numeric  % numeric, alphabetic
]{biblatex}                  % bib management. %bibtex
\usepackage{minitoc}         % table of contents per chapter.
%\usepackage{titlesec}        % change titles size.
\usepackage{algorithm}       % algorithm block.
\usepackage{algcompatible}
\usepackage{algpseudocode}   % style for (autoimported) package algorithmicx.
\usepackage{float}           % float management.
\usepackage[toc,page]{appendix}  % appendix.
\usepackage{tcolorbox}
%\usepackage{minted}
\usepackage{tikz}            % flow chart.
%\usepackage{tocvsec2}        % numbering chapter fix.
\usepackage{enumitem}        % enums.
%\usepackage{subcaption}      % needed by nested figures.
%\usepackage{showframe}      % DEBUG: shows page frames.
\usepackage{xcolor}

%% ***************************************************************
%%    RESOURCES
\input{prooftree.tex}
\input{macros.tex}

%% ***************************************************************
%%    CONFIGURATIONS
\usetikzlibrary{calc,trees,positioning,arrows,chains,shapes.geometric,%
	decorations.pathreplacing,decorations.pathmorphing,shapes,%
	matrix,shapes.symbols}
\tikzset{
	% flow chart
	>=stealth',
	punktchain/.style={
		rectangle,
		rounded corners,
		% fill=black!10,
		draw=black, very thick,
		text width=10em,
		minimum height=3em,
		text centered,
		on chain},
	line/.style={draw, thick, <-},
	element/.style={
		tape,
		top color=white,
		bottom color=blue!50!black!60!,
		minimum width=8em,
		draw=blue!40!black!90, very thick,
		text width=10em,
		minimum height=3.5em,
		text centered,
		on chain},
	every join/.style={->, thick,shorten >=1pt},
	decoration={brace},
	tuborg/.style={decorate},
	tubnode/.style={midway, right=2pt},
}

%% ***************************************************************
%%    OPENING
\title{Workbook \\ Master Degree in Computer Science}
\author{Luca Parolari\footnote{\href{mailto:luca.parolari23@gmail.com}{luca.parolari23@gmail.com}}}


%% ###############################################################
%%                            DOCUMENT
%% ###############################################################
\begin{document}

\maketitle
\tableofcontents

\part{AA 2019-2020}

% *********************
\chapter{Software Analysis and Verification}

\section{Exercises}

\subsection{Exercise 1.7}

\begin{exercise}{(1.7)}
	\label{ex_1_7}
	Prove that the equations of Equation \ref{eq_ex_1_7} define a total function $\calA$ in $\AExp \to (\ST \to \mathrm{Z})$: First argue that it is sufficient to prove that for each $a \in \AExp$ and each $s \in \ST$ there is exactly one value $v \in Z$ such that $\denotA{a}{s} = v$. Next use structural induction on the arithmetic expressions to prove that this is indeed the case.
	
	\begin{equation}
	\label{eq_ex_1_7}
	\begin{split}
	\denotA{n}{s} &= \denotN{n} \\
	\denotA{x}{s} &= s x \\
	\denotA{a_1+a_2}{s} &= \denotA{a_1}{s} + \denotA{a_2}{s} \\
	\end{split}
	\end{equation}
	
	\begin{proof}
		We have to prove that
		\[
		\forall a \in \AExp \itc \forall s \in \ST: \exists v \in \mathrm{Z} \st \denotA{a}{s} = v
		\]
		and we can do it by structural induction on $a$.
		
		\begin{itemize}
			\item Base case $a \equiv n$. $\denotA{n}{s} = \denotN{n}$ holds assuming $\denotN{n}$ is a total function.
			\item Base case $a \equiv x$. $\denotA{x}{s} = s x$ holds by definition of $s$ ($\fund{s}{\Var}{\Natural}$).
			\item Inductive case $a \equiv a_1 + a_2$. We can apply the definition for this case and we obtain
			\[
			\forall a_1, a_2 \in \AExp \itc \forall s \in \ST: \exists v \in \mathrm{Z} \st \denotA{a_1 + a_2}{s} = v
			\]
			and this, by structural induction and because $+$ is a total function, holds.
		\end{itemize}
	\end{proof}
	
\end{exercise}

\subsection{Exercise 1.8}

\begin{exercise}{(1.8)}
	Assume that $s\ x = 3$ and determine $\denotB{\notop(x = 1)}{s}$.
	
	\begin{proof}
		We have to apply rules from our semantics in order to get the result.
		
		TODO
	\end{proof}
\end{exercise}

\subsection{Exercise 1.9}

\begin{exercise}{(1.9)}
	Prove that the denotation semantic function for booleans $\calB$, $\fund{\calB}{\BExp}{(\ST \to \mathrm{T})}$ is total.
	
	\begin{proof}
		Identical to \ref{ex_1_7}
	\end{proof}
\end{exercise}

\subsection{Exercise 1.9}

\begin{exercise}{(1.9)}
	TODO
\end{exercise}

\subsection{Exercise 1.12}

\begin{exercise}
	Let $s$ and $s'$ be two states satisfying that $s x = s' x$ for all $x$ in $FV(b)$. Prove that $\denotB{b}{s} = \denotB{b}{s'}$. $\FV$ is defined in Definition \ref{ex_1_12_fv}.
	
	\begin{proof}
		We assume that this sentences holds for $\calA$. (This is proved in \ref{ex_1_12_lemma}). We have to prove that
		\[
		\forall b \in \BExp \itc \forall s, s' \in \ST \itc \forall x \in \FV(a) \itc sx = s'x \Rightarrow \denotB{b}{s} = \denotB{b}{s'}
		\]
		
	\end{proof}
	
\end{exercise}

\begin{definition}
	\label{ex_1_12_fv}
	Define FV \todo{define FV}
\end{definition}

\begin{lemma}
	\label{ex_1_12_lemma}
	\[
	\forall a \in \AExp \itc \forall s, s' \in \ST \itc \forall x \in \FV(a) \itc sx = s'x \Rightarrow \denotA{a}{s} = \denotA{a}{s'}
	\]
	
	\begin{proof}
		We have to prove that
		\[
		\forall a \in \AExp \itc \forall s, s' \in \ST \st sx = s'x \Rightarrow \denotA{b}{s} = \denotA{b}{s'}
		\]
		and we can do it by structural induction on $b$.
		
		\begin{itemize}
			\item Base case $a \equiv n$. It's trivial to note that $\denotA{n}{s} = \denotN{n} = n = n = \denotN{n} = \denotB{n}{s'}$.
			\item Base case $a \equiv x$. In this case we have
		\end{itemize}
	\end{proof}
\end{lemma}

\end{document}
